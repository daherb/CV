%!TEX encoding = UTF-8 Unicode
% LaTeX Template for CV/Resume
%
% This template has been downloaded and modified from:
% https://github.com/posquit0/Awesome-CV
%
% Original Author:
% Claud D. Park <posquit0.bj@gmail.com>
% http://www.posquit0.com
%
% Modified (March 2019) by:
% Jessica A. Rick <jrick@uwyo.edu>
% http://www.jessicarick.com
%
%
%-------------------------------------------------------------------------------
% CONFIGURATIONS
%-------------------------------------------------------------------------------
% A4 paper size by default, use 'letterpaper' for US letter
\documentclass[11pt, a4]{academic-cv}
% Configure page margins with geometry
\geometry{left=0.75in, top=0.75in, right=0.75in, bottom=0.75in, footskip=.5cm}


%% % Specify the location of the included fonts
%% \fontdir[fonts/]


% If you would like to change the social information separator from a pipe (|) to something else
\renewcommand{\acvHeaderSocialSep}{\quad\textbar\quad}


\usepackage{hyperref}
\usepackage{csquotes}
\usepackage{adjustbox}


%-------------------------------------------------------------------------------
% PERSONAL INFORMATION
% Comment any of the lines below if they are not required
%-------------------------------------------------------------------------------
\name{Herbert Lange}{}
\position{Doctor of Philosophy {\enskip\cdotp\enskip} Computer Science}
\address{Munich, Germany}
%\mobile{+1 000-000-0000}
\email{contact (AT) hackerbrau.se}
\homepage{https://hackerbrau.se}
\github{daherb}
\linkedin{herbert-l-2474638a/}

\twitter{@pietaetskirsche}


% \extrainfo{extra informations}
%-------------------------------------------------------------------------------
\begin{document}
% Print the header with above personal information
% Give optional argument to change alignment(C: center, L: left, R: right)
\makecvheader
% Print the footer with 3 arguments(<left>, <center>, <right>)
% Leave any of these blank if they are not needed
\makecvfooter
{ October 2021 }
{ Herbert Lange ~~~·~~~ Curriculum Vit\ae}
{\thepage}


%-------------------------------------------------------------------------------
% CV/RESUME CONTENT
% Each section is imported separately, open each file in turn to modify content
% Comment out any sections that are not appropriate for your CV
%-------------------------------------------------------------------------------
%-------------------------------------------------------------------------------
% SECTION TITLE
%-------------------------------------------------------------------------------
\vspace{-10px}
\cvsection{University Degrees}

\begin{cventries}
\cventry
{PhD in computer science} % Degree
{\adjustbox{valign=t}{\includegraphics[width=30px]{icons/gu.png}} University of Gothenburg} % Institution
{Gothenburg, Sweden} % Location
{March 2021} % Date(s)
{Thesis: ``Learning Language (with) Grammars: From Teaching Latin to Learning Domain-Specific Grammars''}

\cventry
{Licentiate in computer science} % Degree
{\adjustbox{valign=t}{\includegraphics[width=30px]{icons/gu.png}} University of Gothenburg} % Institution
{Gothenburg, Sweden} % Location
{November 2018} % Date(s)
{Thesis: ``Computer-Assisted Language Learning with Grammars. A Case Study on Latin Learning''}

\cventry
{Master's in computational linguistics} % Degree
{\adjustbox{valign=t}{\includegraphics[width=30px]{icons/lmu.png}} Ludwig-Maximilians-University} % Institution
{Munich, Germany} % Location
{July 2014} % Date(s)
{Thesis: ``Implementierung einer Lateingrammatik im Grammatical Framework''}


\end{cventries}
\vspace{-10px}
\cvsection{University Edudation}

\begin{cventries}
\cventry
{PhD studies in computer science} % Degree
{\adjustbox{valign=t}{\includegraphics[width=30px]{icons/gu.png}} University of Gothenburg} % Institution
{Gothenburg, Sweden} % Location
{August 2015 -- Semptember 2020} % Date(s)
{
\begin{cvitems} % Description(s) bullet points
\item Supervisor: Peter Ljunglöf
\item Co-supervisor: Koen Claessen
\item Internal examinor: Aarne Ranta
\end{cvitems}
}

\cventry
{Erasmus stay} % Degree
{\adjustbox{valign=t}{\includegraphics[width=30px]{icons/tcd.jpg}} Trinity College Dublin} % Institution
{Dublin, Ireland} % Location
{Semptember 2010 -- May 2011} % Date(s)
{
}

\cventry
{Master's in computational linguistics} % Degree
{\adjustbox{valign=t}{\includegraphics[width=30px]{icons/lmu.png}} Ludwig-Maximilians-University} % Institution
{Munich, Germany} % Location
{October 2008 -- Semptember 2014} % Date(s)
{
\begin{cvitems} % Description(s) bullet points
\item Supervisor: Hans Leiss
\end{cvitems}
}

\end{cventries}
\vspace{-10px}
\cvsection{Job Experience}

\begin{cventries}
\cventry
{Researcher} % Degree
{\adjustbox{valign=t}{\includegraphics[width=30px]{icons/ids.png}} Leibniz-Institute for the German Language} % Institution
{Mannheim, Germany} % Location
{March 2022 -- now} % Date(s)
{
Development of a long-term archiving system for linguistic research data
}

\cventry
{Researcher} % Degree
{\adjustbox{valign=t}{\includegraphics[width=30px]{icons/unihh.jpg}} University of Hamburg} % Institution
{Hamburg, Germany} % Location
{June 2021 -- November 2022} % Date(s)
{
Development of quality assurance solutions for audio-visual corpora
}

\cventry
{Researcher} % Degree
{\adjustbox{valign=t}{\includegraphics[width=30px]{icons/gu.png}} University of Gothenburg} % Institution
{Gothenburg, Sweden} % Location
{August 2015 -- Semptember 2020} % Date(s)
{
PhD studies in computer science with research focus on computational linguistics
}

\cventry
{Software developer and system administrator} % Degree
{self-employed} % Institution
{Munich, Germany} % Location
{April 2015 -- May 2015} % Date(s)
{
Development of a monitoring solution ofr a segment of the university network at Ludwig-Maximilians-University, Munich
}

\cventry
{Software engineer} % Degree
{\adjustbox{valign=t}{\includegraphics[width=30px]{icons/glanos.png}} Glanos GmbH} % Institution
{Munich, Germany} % Location
{November 2014 -- January 2015} % Date(s)
{
Java backend development handling XML/JSON data and manual data annotation and analysis for Named Entity Recognition
}

\cventry
{Student assistant} % Degree
{\adjustbox{valign=t}{\includegraphics[width=30px]{icons/lmu.png}} Institute for German Philology, Ludwig-Maximilians-University} % Institution
{Munich, Germany} % Location
{November 2011 -- December 2011} % Date(s)
{
Data annotation for online publication of the " Deutschen Verfasserlexikons der Frühen Neuzeit"
}

\cventry
{Student assistant} % Degree
{\adjustbox{valign=t}{\includegraphics[width=30px]{icons/lmu.png}} IT-Zentrum der Sprach- und Literaturwissenschaften, Ludwig-Maximilians-University} % Institution
{Munich, Germany} % Location
{February 2009 -- October 2014} % Date(s)
{
System administration for a segment of the university network consisting of Linux/Mac/Windows servers and Windows/Mac clients
}

\end{cventries}
\vspace{-10px}
\cvsection{Publications}

\begin{cvpubs}
\cvpub{Herbert Lange and Jocelyn Aznar (2022): ``RefCo and its Checker: Improving Language Documentation Corpora’s Reusability Through a Semi-Automatic Review Process'', Proceedings of the 13th Language Resources and Evaluation Conference (LREC), Marseille, France, European Language Resources Association, 2721--2729, \url{https://aclanthology.org/2022.lrec-1.291}, Published.}

\cvpub{Herbert Lange and Peter Ljunglöf (2021): ``Learning Domain-Specific Grammars From a Small Number of Examples'', Natural Language Processing in Artificial Intelligence, Studies in Computational Intelligence (SCI) (939), 105-138, Springer International Publishing, Cham, Switzerland, \url{https://doi.org/10.1007/978-3-030-63787-3_4}, Published.}

\cvpub{\emph{Proceedings of the 9th Workshop on Natural Language Processing for Computer Assisted Language Learning} (2020); David Alfter, Elena Volodina, Ildikó Pilán, Herbert Lange and Lars Borin (eds.), Linköping University Electronic Press, Linköping, Sweden, Linköping Electronic Conference Proceedings (175), 45 pages, \url{https://doi.org/10.3384/ecp20175}, Published.}
\cvpub{Herbert Lange (2020): ``Learning Language (with) Grammars: From Teaching Latin to Learning Domain-Specific Grammars'', PhD thesis, Department of Computer Science and Engineering, University of Gothenburg, Gothenburg, Sweden, \url{http://hdl.handle.net/2077/65453}, Published.}

\cvpub{Herbert Lange and Peter Ljunglöf (2020): ``Learning Domain-specific Grammars from a Small Number of Examples'', Proceedings of the 12th International Conference on Agents and Artificial Intelligence - Volume 1: NLPinAI, Valetta, Malta, INSTICC. SciTePress, 422–430, \url{https://doi.org/10.5220/0009371304220430}, Published.}

\cvpub{\emph{Proceedings of the 8th Workshop on Natural Language Processing for Computer Assisted Language Learning} (2019); David Alfter, Elena Volodina, Lars Borin, Ildikó Pilán and Herbert Lange (eds.), Linköping University Electronic Press, Linköping, Sweden, NEALT Proceedings Series  (39), 99 pages, Published.}
\cvpub{Herbert Lange (2021): ``An Open-Source Computational Latin Grammar: Overview and Evaluation'', Proceedings of the 20th International Colloquium on Latin Linguistics (ICLL 2019), Madrid, Spain, Ediciones Clásicas, 559-578, Published.}

\cvpub{Herbert Lange and Peter Ljunglöf (2020): ``Demonstrating the MUSTE Language Learning Environment'', Proceedings of the 7th Workshop on NLP for Computer Assisted Language Learning (NLP4CALL 2018) at SLTC, 7th November 2018, Stockholm, Sweden, Linköping University Electronic Press, 41–46, \url{https://www.aclweb.org/anthology/W18-7105}, Published.}

\cvpub{Herbert Lange (2018): ``Computer-Assisted Language Learning with Grammars. A Case Study on Latin Learning'', Licentiate thesis, Department of Computer Science and Engineering, University of Gothenburg, Gothenburg, Sweden, \url{https://gup.ub.gu.se/file/207536}, Published.}

\cvpub{Herbert Lange and Peter Ljunglöf (2018): ``Putting Control into Language Learning'', Proceedings of the Sixth International Workshop on Controlled Natural Languages, Maynooth, Ireland, IOS Press, Frontiers in Artificial Intelligence and Applications (304), 61-70, \url{https://doi.org/10.3233/978-1-61499-904-1-61}, Published.}

\cvpub{Herbert Lange and Peter Ljunglöf (2018): ``MULLE: A Grammar-based Latin Language Learning Tool to Supplement the Classroom Setting'', Proceedings of the 5th Workshop on Natural Language Processing Techniques for Educational Applications (NLPTEA '18) at ACL, Melbourne, Australia, Association for Computational Linguistics, 108-112, \url{https://doi.org/10.18653/v1/W18-3715}, \url{http://aclweb.org/anthology/W18-3715}, Published.}

\cvpub{Herbert Lange (2017): ``Implementation of a Latin Grammar in Grammatical Framework'', Proceedings of the 2nd International Conference on Digital Access to Textual Cultural Heritage (DATeCH2017), Göttingen, Germany, Association for Computing Machinery, 97-102, \url{https://doi.org/10.1145/3078081.3078108}, Published.}

\cvpub{Herbert Lange (2013): ``Erstellen einer Grammatik für das Lateinische im "Grammatical Framework"'', Master's thesis, Centrum für Informations- und Sprachverarbeitung, Ludwig-Maximilians-University, Munich, Germany, Published.}


\end{cvpubs}
\vspace{-10px}
\cvsection{Presentations}

\begin{cvpubs}
\cvpub{Herbert Lange (2022): ``The Hith-Hikers Guide to Artificial Intelligence - BBC Basic Version'', Updateringar - Update Computer Club Uppsala, Online, Talk: \url{https://wiki.dfupdate.se/projekt:updateringar}}

\cvpub{Herbert Lange (2022): ``Semi-automatic quality assurance for audiovisual corpus data'', CLT Seminar. University of Gothenburg, Gothenburg, Sweden, Seminar Talk}

\cvpub{Herbert Lange (2022): ``Demonstrating an Automatic Gloss Checker for Annotated Corpora'', Language Documentation and Archiving during the Decade of Indigenous LanguagesConference and training sessions, Berlin, Germany, Conference talk}

\cvpub{Herbert Lange and Jocelyn Aznar (2022): ``Training Session: Improving Corpus Quality in Language Documentation : Introduction to QUEST and the RefCo process'', Language Documentation and Archiving during the Decade of Indigenous LanguagesConference and training sessions, Online, Conference talk}

\cvpub{Herbert Lange and Jocelyn Aznar (2022): ``RefCo and its Checker: Improving Language Documentation Corpora's Reusability Through a Semi-Automatic Review Process'', Language Resoures and Evaluation Conference (LREC), Marseille, France, Conference talk}

\cvpub{Herbert Lange (2022): ``MULLE: A grammar-based language learning tool'', Seminar on Language Technology for Education in the South African languages, Online, Seminar Talk}

\cvpub{Herbert Lange (2020): ``Type theory and meaning in linguistics'', rC3 - remote Chaos Experience, Online, Talk: \url{https://media.ccc.de/v/rc3-232856-type_theory_and_meaning_in_linguistics}}

\cvpub{Herbert Lange (2020): ``A Type-Theoretic Approach to Generating Pictures and Descriptions'', 8th Swedish Language Technology Conference, Gothenburg, Sweden, Conference talk}

\cvpub{Herbert Lange (2020): ``Learning Domain-specific Grammars from Examples'', CLASP Seminar, University of Gothenburg, Gothenburg, Sweden, Seminar Talk}

\cvpub{Herbert Lange (2020): ``Learning Domain-specific Grammars from a Small Number of Examples'', Special Session NLPinAI, 12th International Conference on Agents and Artificial Intelligence, Valetta, Malta, Conference talk}

\cvpub{Herbert Lange (2020): ``Using Dependent Types in GF'', Functional Programming Winter Meeting, Chalmers University of Technology, Gothenburg, Sweden, Seminar Talk}

\cvpub{Herbert Lange (2019): ``Empirical Evaluation of a Computational Latin Grammar'', 20th International Colloquium on Latin Linguistics, Las Palmas de Gran Canaria, Canary Islands, Conference talk}

\cvpub{Herbert Lange (2019): ``A short history of end-user programming'', Update mini conference, Uppsala, Sweden, Talk}

\cvpub{Herbert Lange (2019): ``Computer-Assisted Language Learning for Latin'', Latin Seminar, Uppsala University, Uppsala, Sweden, Seminar Talk}

\cvpub{Herbert Lange (2019): ``Restricting Grammars to Reduce Ambiguity'', Functional Programming Seminar, Chalmers University of Technology, Gothenburg, Sweden, Seminar Talk}

\cvpub{Herbert Lange (2019): ``MULLE for Latin: Computer-Generated Translation Exercises for Latin'', Workshop on Digital Approaches to Teaching Historical Languages, Berlin, Germany, Conference talk: \url{https://www.projekte.hu-berlin.de/en/callidus-en/DAtTeL-workshop/digital-approaches-to-teaching-historical-languages-dattel}}

\cvpub{Herbert Lange and Peter Ljunglöf (2018): ``Demonstrating the MUSTE Language Learning Environment'', 7th Workshop on Natural Language Processing for Computer-Assisted Language Learning at the Swedish Language Technology Conference (SLTC), Stockholm, Sweden, Poster}

\cvpub{Herbert Lange (2018): ``MULLE: A grammar-based Latin language learning tool to supplement the classroom setting'', 5th Workshop on Natural Language Processing Techniques for Educational Applications, Melbourne, Australia, Poster}

\cvpub{Herbert Lange (2018): ``Computational Linguistics vs. Natural Language Processing - A bit of a rant'', Free Society Conference and Nordic Summit (FSCONS), Oslo, Norway, Lightning Talk: \url{https://youtu.be/Xrb3ULik1vc?t=3127}}

\cvpub{Herbert Lange (2018): ``Let's talk about Old Computer; Or: Why old computers are cool, why we should care, and stuff I discovered'', Free Society Conference and Nordic Summit (FSCONS), Oslo, Norway, Lightning Talk: \url{https://youtu.be/_C5QUuU2vic?t=1386}}

\cvpub{Herbert Lange (2017): ``Implementation of a Latin Grammar in Grammatical Framework'', 2nd International Conference on Digital Access to Textual Cultural Heritage, Göttingen, Germany, Conference talk}

\cvpub{Herbert Lange (2017): ``A Latin Language Learning Application'', Latin Seminar, University of Gothenburg, Gothenburg, Sweden, Seminar Talk}

\cvpub{Herbert Lange (2017): ``From Word-based text editing to language learning'', Dublin Computational Linguistics Research Seminar, Trinity College Dublin, Dublin, Ireland, Seminar Talk}

\cvpub{Herbert Lange (2017): ``From Word-based text editing to language learning'', Postgraduate Seminar in Computer Science, National University of Ireland, Maynooth, Ireland, Seminar Talk}

\cvpub{Herbert Lange (2017): ``MUSTE - Behind the scenes'', REMU Research Seminar, University of Gothenburg, Gothenburg, Sweden, Seminar Talk}

\cvpub{Herbert Lange (2016): ``Implementation of a Latin Grammar in Grammatical Framework'', 6th Swedish Language Technology Conference (SLTC), Umeå, Sweden, Poster}

\cvpub{Herbert Lange (2016): ``SHRDLU - Ein Programm das natürliche Sprache versteht'', Vintage Computing Festival Berlin (VCFB), Berlin, Germany, Talk: \url{https://media.ccc.de/v/vcfb2016_-_52_-_en_-_medientheater_-_201610031530_-_shrdlu_-_herbert_lange}}

\cvpub{Herbert Lange (2015): ``Vintage Computing'', Free Society Conference and Nordic Summit (FSCONS), Gothenburg, Sweden, Lightning Talk: \url{https://youtu.be/8mSVMY74sOY?t=660}}

\cvpub{Herbert Lange (2013): ``Grammatical Framework'', Mehrvorträgewagen, muCCC Munich Hackerspace, Munich, Germany, Talk}

\end{cvpubs}
\vspace{-10px}
\cvsection{Teaching Experience}

\begin{cventries}
\cventry
{Chalmers University of Technology and University of Gothenburg}
{Introduction to concurrent programming}
{Teaching Assistant}
{}
{Spring 2019}

\cventry
{Chalmers University of Technology and University of Gothenburg}
{Artificial Intelligence}
{Teaching Assistant}
{}
{Spring 2018, Spring 2019}

\cventry
{University of Gothenburg}
{Computational Synax}
{Teaching Assistant}
{}
{Spring 2017, Spring 2018, Spring 2019, Spring 2020}

\cventry
{Grammatical Framework Summer School, Rīga, Latvia}
{Tutorial: GF for Python programmers}
{Tutor}
{}
{August 2017}

\cventry
{Chalmers University of Technology and University of Gothenburg}
{Functional Programming}
{Teaching Assistant}
{}
{Autumn 2016, Autumn 2017, Autumn 2018, Autumn 2019}

\cventry
{Chalmers University of Technology and University of Gothenburg}
{Databases}
{Teaching Assistant}
{}
{Autumn 2015, Spring 2016, Autumn 2016, Spring 2017, Autumn 2017, Autumn 2018}

\cventry
{TaCoS, German Student Conference for Computational Linguistics), Munich, Germany}
{Tutorial: Introduction to Grammatical Framework}
{Tutor}
{}
{May 2015}

\cventry
{IT-Zentrum der Sprach- und Literaturwissenschaften, Ludwig-Maximilians-University}
{Seminar: Scientific writing in LaTeX}
{Tutor}
{}
{February 2014}

\cventry
{Centre for Information and Language Processing, Ludwig-Maximilians-University Munich}
{Logic and model-theoretic Semantics}
{Teaching Assistant}
{}
{Summer 2012}

\cventry
{Centre for Information and Language Processing, Ludwig-Maximilians-University Munich}
{Corpus and Unix tools}
{Teaching Assistant}
{}
{Summer 2011}

\end{cventries}
\vspace{-10px}
\cvsection{Grants}

\begin{cvhonors}
\cvhonor
{Grammatical Framework Summer School, Gozo, Malta}
{\newline
Travel grant, Centre for Language Technology, University of Gothenburg}
{}
{2015}

\cvhonor
{Grammatical Framework Summer School, Frauenchiemsee, Germany}
{\newline
Travel grant, Volkswagenstiftung}
{}
{2013}

\cvhonor
{Trinity College, Dublin, Ireland}
{\newline
Erasmus grant, European Union}
{}
{2010}
\end{cvhonors}
\vspace{-10px}
\cvsection{Peer Review}

\begin{cvhonors}
\cvhonor {11th Workshop on Natural Language Processing for Computer-Assisted Language Learning}
{}
{}
{2022}

\cvhonor {10th Workshop on Natural Language Processing for Computer-Assisted Language Learning}
{}
{}
{2021}

\cvhonor {7th International Workshop on Controlled Natural Languages}
{}
{}
{2021}

\cvhonor {8th Swedish Language Technology Conference post-proceedings}
{}
{}
{2021}

\cvhonor {9th Workshop on Natural Language Processing for Computer-Assisted Language Learning}
{}
{}
{2020}

\cvhonor {8th Workshop on Natural Language Processing for Computer-Assisted Language Learning}
{}
{}
{2019}

\cvhonor {5th Workshop on Natural Language Processing Techniques for Educational Applications}
{}
{}
{2018}
\end{cvhonors}
\vspace{-10px}
\cvsection{Conference Involvement}

\begin{cvhonors}
\cvhonor {9th Workshop on Natural Language Processing for Computer-Assisted Language Learning}
{Co-Organizer}
{Gothenburg, Sweden}
{2020}

\cvhonor {8th Workshop on Natural Language Processing for Computer-Assisted Language Learning}
{Co-Organizer}
{Turku, Finland}
{2019}

\cvhonor {European Summerschool in Logic, Language and Information Student Session}
{Co-Chair Logic and Computation}
{Toulouse, France}
{2017}
\end{cvhonors}
\vspace{-10px}
\cvsection{Other Involvement}

\begin{cvhonors}
\cvhonor {Graduate Students' Council, University of Gothenburg}
{}
{}
{2018 -- 2020}

\cvhonor {PhD Council, Department for Computer Science and Engineering, Chalmers University of Technology and University of Gothenburg}
{}
{}
{2016 -- 2020}

\cvhonor {Student Council, Centre for Information and Language Processing, Ludwig-Maximilians-University Munich}
{}
{}
{2012 -- 2013}
\end{cvhonors}
\vspace{-10px}
\cvsection{Skills}

Five levels: Basic knowledge (1), Basic experience (2), Professional experience (3), Expert (4), Guru (5)

\cvsubsection{Programming language}
\begin{cvskills}
\cvskill
{}
{Agda}
{Basic knowledge}

\cvskill
{}
{Basic}
{Basic experience}

\cvskill
{}
{C/C++}
{Basic experience}

\cvskill
{}
{Coq}
{Basic knowledge}

\cvskill
{}
{Erlang}
{Basic knowledge}

\cvskill
{}
{Grammatical Framework}
{Expert}

\cvskill
{}
{Haskell}
{Expert}

\cvskill
{}
{HTML/CSS}
{Professional Experience}

\cvskill
{}
{Java}
{Professional Experience}

\cvskill
{}
{JavaScript}
{Professional Experience}

\cvskill
{}
{LISP}
{Basic experience}

\cvskill
{}
{Pascal}
{Basic experience}

\cvskill
{}
{Perl}
{Basic experience}

\cvskill
{}
{Prolog}
{Basic experience}

\cvskill
{}
{Python}
{Professional Experience}

\cvskill
{}
{Ruby}
{Basic knowledge}

\cvskill
{}
{SML}
{Basic experience}

\end{cvskills}

\cvsubsection{Operating systems}
\begin{cvskills}
\cvskill
{}
{Atari TOS}
{Basic knowledge}

\cvskill
{}
{Gentoo Linux}
{Expert}

\cvskill
{}
{BeOS/Haiku}
{Basic knowledge}

\cvskill
{}
{FreeBSD/NetBSD}
{Basic experience}

\cvskill
{}
{IRIX}
{Basic experience}

\cvskill
{}
{MacOS Classic}
{Basic experience}

\cvskill
{}
{MacOS X}
{Basic experience}

\cvskill
{}
{MS-DOS}
{Basic experience}

\cvskill
{}
{OpenVMS}
{Basic knowledge}

\cvskill
{}
{OS/2}
{Basic knowledge}

\cvskill
{}
{RISC OS}
{Basic knowledge}

\cvskill
{}
{Windows}
{Professional Experience}

\end{cvskills}

\cvsubsection{Other IT skills}
\begin{cvskills}
\cvskill
{}
{Constraint Programming}
{Basic experience}

\cvskill
{}
{Formal languages}
{Professional Experience}

\cvskill
{}
{GNU make}
{Basic experience}

\cvskill
{}
{Icinga/Nagios}
{Basic experience}

\cvskill
{}
{LaTeX}
{Professional Experience}

\cvskill
{}
{Machine Learning}
{Basic knowledge}

\cvskill
{}
{MS Office}
{Professional Experience}

\cvskill
{}
{Parser combinators}
{Professional Experience}

\cvskill
{}
{Property-based Testing}
{Professional Experience}

\cvskill
{}
{Software-defined Radio}
{Basic experience}

\cvskill
{}
{Databases}
{Professional Experience}

\cvskill
{}
{Version control systems}
{Professional Experience}

\end{cvskills}

\cvsubsection{Language}
\begin{cvskills}
\cvskill
{}
{English}
{Expert}

\cvskill
{}
{Esperanto}
{Basic experience}

\cvskill
{}
{French}
{Basic knowledge}

\cvskill
{}
{German}
{Expert}

\cvskill
{}
{Irish}
{Basic knowledge}

\cvskill
{}
{Italian}
{Basic knowledge}

\cvskill
{}
{Latin}
{Basic experience}

\cvskill
{}
{Medieval German}
{Basic knowledge}

\cvskill
{}
{Spanish}
{Basic knowledge}

\cvskill
{}
{Swedisch}
{Professional Experience}

\end{cvskills}

\cvsubsection{Computational linguistics}
\begin{cvskills}
\cvskill
{}
{Controlled Natural Languages}
{Expert}

\cvskill
{}
{Discourse Representation Theory}
{Basic experience}

\cvskill
{}
{Formal syntax}
{Expert}

\cvskill
{}
{Lexikal semantics}
{Basic experience}

\cvskill
{}
{Montague grammars}
{Professional Experience}

\cvskill
{}
{Resource grammar development}
{Guru}

\cvskill
{}
{Universal Dependencies}
{Basic experience}

\cvskill
{}
{Typ-theoretic semantics}
{Basic experience}

\end{cvskills}

\cvsubsection{Research and generic skills}
\begin{cvskills}
\cvskill
{}
{Data analysis}
{Professional Experience}

\cvskill
{}
{Experimental design in social sciences}
{Basic experience}

\cvskill
{}
{Independent work}
{Expert}

\cvskill
{}
{Problem solving}
{Expert}

\cvskill
{}
{Project management}
{Professional Experience}

\cvskill
{}
{Public presentation}
{Professional Experience}

\cvskill
{}
{Research}
{Expert}

\cvskill
{}
{Research ethics}
{Professional Experience}

\cvskill
{}
{Scientific writing}
{Expert}

\cvskill
{}
{Teaching}
{Professional Experience}

\cvskill
{}
{Team work}
{Professional Experience}

\end{cvskills}


%-------------------------------------------------------------------------------
\end{document}

